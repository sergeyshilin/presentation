\documentclass[11pt,pdf,hyperref={unicode}]{beamer}
\usepackage[english,russian]{babel}
\usepackage{sidecap}
\usepackage{geometry}

% подключаем кириллицу 
\usepackage[T2A]{fontenc}
\usepackage[utf8]{inputenc}

\usepackage{wrapfig}
\usepackage[makeroom]{cancel}
\usepackage{braket}
\usepackage{amsmath}
\usepackage{graphicx}
\usepackage{amssymb}

\usepackage{bookmark}
% \usepackage{etoolbox}

% отключить клавиши навигации
% \setbeamertemplate{navigation symbols}{}

\usetheme{Malmoe}
\usefonttheme[onlymath]{serif}
\usecolortheme{seahorse}

\title[Выделение сообществ в графах \hspace*{1cm} SCVRT2015]{Разработка и реализация моделей и алгоритмов выделения сообществ в графах взаимодействующих объектов}
\author[C. Шилин]{Сергей Шилин \vspace{-0.7cm}}
\date[SCVRT2015]{Международная конференция SCVRT2015 \break 24-27 ноября 2015 г. }
\titlegraphic{\includegraphics[width=2cm]{images/FTI-logo.jpg}\hspace*{4.75cm}~%
   \includegraphics[width=2cm]{images/IFTI-FTI-logotip.jpg}
}

% \makeatletter
% \apptocmd{\beamer@@frametitle}{\bookmark[page=\the\c@page,level=4]{#1}}%
% {\message{** patching of \string\beamer@@frametitle succeeded **}}%
% {\message{** patching of \string\beamer@@frametitle failed **}}%
% \makeatother

\setbeamertemplate{section in toc}[sections numbered]
\setbeamertemplate{subsection in toc}[ball unnumbered]
\setbeamertemplate{itemize items}[ball]
\setbeamertemplate{enumerate items}[ball]


\begin{document}

% титульный слайд
\frame{\titlepage}
\begin{frame}{Содержание}
	\tableofcontents
\end{frame}

\section{Введение} % (fold)
\label{sec:intro}

	\subsection{Постановка задачи} % (fold)
	\label{sub:problem}
		\begin{frame}{Введение: Постановка задачи} 
			\begin{itemize}
				\item Улучшить что-то
				\pause \item Обновить что-то
				\pause \item Сделать круто
			\end{itemize}
		\end{frame}

	\subsection{Парампам} % (fold)
	\label{sub:params}
		\begin{frame}{Введение: Парампам}
			\begin{itemize}
				\item Улучшить что-то
				\pause \item Обновить что-то
				\pause \item Сделать круто
			\end{itemize}
		\end{frame}
	
	% subsection params (end)

	\subsection{Трололо} % (fold)
	\label{sub:trololo}
		\begin{frame}{Введение: Трололо}
			\begin{itemize}
				\item Улучшить что-то
				\pause \item Обновить что-то
				\pause \item Сделать круто
			\end{itemize}
		\end{frame}
	
	% subsection trololo (end)

% section intro (end)

\section{Продолжение} % (fold)
\label{sec:continious}

	\subsection{Постановка задачи} % (fold)
	\label{sub:problem2}
		\begin{frame}{Введение: Постановка задачи} 
			\begin{itemize}
				\item Улучшить что-то
				\pause \item Обновить что-то
				\pause \item Сделать круто
			\end{itemize}
		\end{frame}

	\subsection{Парампам} % (fold)
	\label{sub:params2}
		\begin{frame}{Введение: Парампам}
			\begin{itemize}
				\item Улучшить что-то
				\pause \item Обновить что-то
				\pause \item Сделать круто
			\end{itemize}
		\end{frame}
	
	% subsection params (end)

	\subsection{Трололо} % (fold)
	\label{sub:trololo2}
		\begin{frame}{Введение: Трололо}
			\begin{itemize}
				\item Улучшить что-то
				\pause \item Обновить что-то
				\pause \item Сделать круто
			\end{itemize}
		\end{frame}

% section continious (end)


\end{document}